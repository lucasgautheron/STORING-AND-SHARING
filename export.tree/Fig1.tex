\tikzset{%
  >={Latex[width=2mm,length=2mm]},
            base/.style = {rectangle, rounded corners, draw=black,
                           minimum width=4cm, minimum height=1cm,
                           text centered, font=\sffamily},
  recorder/.style = {base, fill=blue!30},
       meta/.style = {base, fill=red!30},
    annotator/.style = {base, fill=green!30},
          txt/.style = {font=\sffamily, text centered},
         title/.style = {txt, font=\sffamily\Large},
}

% Drawing part, node distance is 1.5 cm and every node
% is prefilled with white background
\begin{tikzpicture}[node distance=5cm,
    every node/.style={fill=white, font=\sffamily}, align=center]
  % Specification of nodes (position, etc.)
%   \node (lena) [recorder] {LENA recorder};
%   \node (babylogger) [recorder, right of=lena, xshift=5em] {BabyLogger};
%   \node (others) [recorder, right of=babylogger] {Other alternatives\\
%   \footnotesize{USB, Olympus...}};

%   \node (lena_software) [classifier, below of=lena, yshift = 5em] {LENA software\\
%   \footnotesize{Speaker type, Adult Word Count,}\\
%   \footnotesize{Child Vocalization Count, Conversational Turn Count}};
  
%   \node [txt, below of=lena_software, yshift=10em] {\large{LENA commercial environment}};

%   \node (vtc) [classifier, below of=babylogger, xshift=8em, yshift = 7em] {Voice Type Classifier (VTC)\\
%   \footnotesize {speech detection, speaker type classification}};
  
%   \node (alice) [classifier, below of=vtc, yshift = 8em] {Automatic LInguistic Unit Count Estimator (ALICE)\\
%   \footnotesize {phoneme, syllable and word counts}};


%   \node (seshat) [annotator, below of=lena_software, xshift=15em, yshift = 2em]  {Seshat\\
%   \footnotesize{web-based annotator}\\
%   \footnotesize{inter-rater reliability}};
   
%   \node (zooniverse) [annotator,right of=seshat]  {Zooniverse\\
%   \footnotesize{crowd-sourced classification tasks}};
   
%   \node (elan) [annotator, left of=seshat]  {ELAN\\
%   \footnotesize{annotation software}};
   
%   \node (das) [annotator, below of=elan, yshift = 10em]  {ACLEW DAS\\
%   \footnotesize{annotation scheme}};

%     \node (recorders) [title,right of=others] {Recording device};
%     \node (classifiers) [title,below of=recorders, yshift=3em] {Automatic annotation};
%     \node (annotators) [title,below of=classifiers, yshift=3em] {Manual annotation};


     
%   % Specification of lines between nodes specified above
%   % with aditional nodes for description 
%   \draw[->]             (lena) -- (lena_software);
%   \draw[->]             (vtc) -- (alice);
%   \draw[->]             (elan) -- (das);

% \draw [draw=black,dashed] ($(lena.north west) + (-2,0.5)$) rectangle ($(lena_software.south east) + (1,-1.5)$);

\node (media) [recorder] {
\Large \textbf{Media} \normalsize \\
($\sim 10^2$ to $10^4$ hours) \\
\framebox{
    {\begin{varwidth}{\linewidth}\begin{itemize}
        \item Audio (up to 24 hours \\ per recording)
        \item Video (up to 30 minutes \\ per recording)
        \item Accelerometer data (xyz)
        \item etc.
    \end{itemize}\end{varwidth}}
}};

\node (annotations) [annotator,shape=rectangle,draw,right of=media] {
\Large \textbf{Annotations} \normalsize \\
($\sim 10^5$ to $10^7$ segments)\\ \\
            \begin{tabular}{l r}
             \textbf{manual} & \textbf{automated}\\
              \multicolumn{2}{c}{who-speaks-when} \\
              \multicolumn{2}{c}{linguistic units} \\
              \multicolumn{2}{c}{vocal maturity} \\
               speech directedness &  \\
               transcriptions &  \\

            \end{tabular}
        };
        
\node (metadata) at ($(media)!0.5!(annotations)-(0,3)$) [meta,shape=rectangle,draw] {
\Large \textbf{Metadata} \normalsize \\ \\
            \framebox{
    {\begin{varwidth}{\linewidth}\begin{itemize}
        \item Recordings date and time, type of device, etc.
        \item Child date of birth, gender, normativity, etc.
        \item Socio-economic status, location, language(s), household size, etc.
        \item Questionnaires
    \end{itemize}\end{varwidth}}
}};
        


  \end{tikzpicture}
  
